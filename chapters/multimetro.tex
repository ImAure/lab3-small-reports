\chapter{Misura di resistenze con un multimetro digitale}\label{ch:mult}
    Tra le esperienze svolte con il multimetro digitale riporto la misura delle resistenze di alcuni materiali, tra cui anelli metallici, il corpo umano e alcuni resistori.

    Ai resistori dedico una sezione più approfondita in quanto ho preso \num{50} misure su resistori distinti---ma teoricamente con resistenza uguale---per verificare la distribuzione delle misure di resistenza.

    \section{Il multimetro}
        Lo strumento utilizzato per tutta l'esperienza è un multimetro digitale della serie \emph{DVM841} della \emph{Velleman\textsuperscript{\textregistered}} \cite{velleman-dvm841}. Il multimetro è in grado di misurare tensione continua e alternata, corrente continua e alternata, resistenza, frequenza e temperatura. La risoluzione del multimetro è di \num{2000} punti, il che significa che il display può visualizzare fino a \num{1999} unità.
    \section{Resistori}
        Il kit presenta \num{50} resistori distinti il cui codice colore restituisce un valore\footnote{Lo si può dedurre da qualunque legenda fedele allo standard IEC 60062.} teorico di \SI{820}{\ohm}.