\chapter[Implementazione Numerica della Formula di Bethe--Bloch][Formula di Bethe--Bloch]{Implementazione Numerica della Formula di Bethe--Bloch}\label{ch:bet}

\section{Il modello}
    Come descritto dal Particle Data Group \cite{PhysRevD-110-030001}, la formula di Bethe--Bloch è un modello sperimentale che descrive la perdita di energia di particelle cariche pesanti di media energia---come protoni e particelle $\alpha$---nella materia:
    \begin{equation}
        \left\langle -\dv{E}{x}\right\rangle
        = Kz^2 \frac{Z}{A}\frac{1}{\beta^2}\bqty{\frac{1}{2}\log\frac{2 m_e c^2 \beta^2 \gamma^2 W_\text{max}}{I^2} - \beta^2 - \frac{\delta\pqty{\beta\gamma}}{2}}
        \mycomma
        \label{eq:bet:bethe-bloch-1}
    \end{equation}
    dove $\beta$ e $\gamma$ sono le usuali quantità relativistiche mentre il resto dei simboli sono esplicitati in \tabref{tab:bet:costanti}.
    \begin{table}
        \footnotesize
        \centering
        \begin{tabular*}{\textwidth}{@{\extracolsep{\fill}}lll}\hline\rule{0pt}{8pt}%
    Simbolo & Definizione & Valore o unità di misura\\[0.5pt]
    \hline\hline\rule{0pt}{9pt}%
    $m_e c^2$ & massa a riposo dell'elettrone $\times c^2$ & \SI{ 0.51099895000(15)}{\mega\eV}\\
    $r_e$ & raggio classico dell'elettrone $ e^2/4\pi \epsilon_0 m_e c^2$ & \SI{2.8179403227(19)}{\femto\meter}\\
    $N_\text{A}$ & numero di Avogadro & \SI{ 6.02214076e+23}{mol^{-1}}\\[2pt]
    \hline\rule{0pt}{9pt}%
    $\rho$ & densità & \unit{\g\,\centi\meter^{-3}}\\
    $x$ & massa per unità di area & \unit{\gram\,\centi\meter^{-2}}\\ 
    $M$ & massa della particella incidente & \unit{\mega\eV \,\mathit{c}^{-2}}\\
    $E$ & energia della particella incidente $\gamma M c^2$ & \unit{\mega\eV}\\
    $W_\text{max}$ & massima energia trasferibile per collisioni & \unit{\mega\eV}\\
    $z$ & numero di carica della particella incidente & \\
    $Z$ & numreo atomico del bersaglio & \\
    $A$ & numero di massa atomica del bersaglio & \\
    $K$ & $4\pi N_\text{A} r_e^2 m_e c^2$ & \SI{0.307075}{\mega\eV\,\mol^{-1}\,\centi\meter^2}\\
    $I$ & energia media di eccitazione & \unit{\eV}\\
    $\delta\pqty{\beta\gamma}$ & correzione di ionizzazione & \\[1pt]
    \hline
\end{tabular*}
        \caption{Notazione e unità di misura per la formula di Bethe--Bloch. Si tratta di un riassunto della tabella del PDG \cite{PhysRevD-110-030001}.}
        \label{tab:bet:costanti}
    \end{table}
    La perdita di energia media data dalla \eqref{eq:bet:bethe-bloch-1} è misuarta in \unit{\mega\eV\,\gram^{-1}\,\centi\meter^2} ma può essere portata in \unit{\mega\eV\,\centi\meter^{-1}} moltiplicando entrambi i membri per la densità volumica $\rho$ del bersaglio misurata in \unit{\gram\,\centi\meter^{-3}}. La quantità $W_\text{max}$ è la massima energia che una particella carica può cedere a un elettrone e si esprime come
    \begin{equation}
        W_\text{max} = \frac{2 m_e c^2 \beta^2 \gamma^2}{1 + \gamma\, m_e / M + (m_e / M)^2}
        \myperiod 
        \label{eq:bet:wmax}
    \end{equation}
    La \eqref{eq:bet:bethe-bloch-1} e la \eqref{eq:bet:wmax} sono valide nell'approssimazione $\num{0.1} \lesssim \beta\gamma \lesssim \num{1000}$ poiché al limite inferiore la velocità del proiettile diventa confrontabile con la ``velocità'' degli elettroni atomici mentre al limite superiore gli effetti radiativi non sono più trascurabili.

\section{La simulazione}
    Ho scelto di simulare la perdita di energia di una particella $\alpha$ a \SI{5}{\mega\eV} attraverso un foglio di alluminio di spessore\footnote{Dopo aver provato con uno spessore di \SI{0.016}{\milli\meter} e aver constatato che la curva risultava tagliata---la particella non cedeva tutta l'energia all'alluminio---ho scelto di aumentare di poco lo spessore per il gusto di un grafico più completo.} \SI{0.018}{\milli\meter}. Per realizzare la simulazione ho scritto un codice in C che implementa la \eqref{eq:bet:bethe-bloch-1} in modo approssimato.
    \subsection{Descrizione del codice e dei calcoli}
        Il programma inizia richiedendo da tastiera il numero di intervalli $N$ in cui suddividere lo spessore del materiale; gli altri dati sono tutti precompilati come costanti. 
        
        Assumiamo che in ciascun intervallo di spessore\footnote{Ricordo che in questo caso la $x$ non indica una distanza.} $\dd{s}$ tutte le quantità variabili di nostro interesse siano costanti---velocità, energia, perdita di energia \myetc. Dalla teoria della relatività ristretta scriviamo per l'$n$-esimo intervallo
        \begin{align*}   
                E_n &= T_n + M c^2 \mycomma \\
                E_n &= \gamma_n M c^2 \mycomma
        \end{align*}
        dove $M$ è la massa a riposo della particella incidente, $T_n$ la sua energia cinetica ed $E_n$ la sua energia totale. Note queste ultime due quantità, è possibile calcolare il fattore di Lorentz $\gamma_n$ e $\beta_n^2$ come
        \begin{equation*}
            \gamma_n = \frac{T_n + M c^2}{M c^2}
            \,,
            \quad
            \beta_n^2 = 1 - \frac{1}{\gamma_n^2}
            \myperiod
        \end{equation*}
        Le quantità appena descritte sono calcolate attraverso il codice nelle righe \lnref{ln:bet:T-start}--\lnref{ln:bet:T-end}.

        Noti $\gamma_n$ e $\beta_n$, è possibile calcolare la perdita di energia media per unità di lunghezza attraverso la \eqref{eq:bet:bethe-bloch-1} moltiplicata per $\rho$. Possiamo in particolare calcolare l'energia cinetica con cui la particella $\alpha$ entra nello strato successivo, $T_{n + 1}$ come
        \begin{equation*}
            T_{n + 1} = T_n + \dd{T_n} = T_n + \rho\, {\left\langle - \dv{E}{x} \right\rangle}_n \dd{s}
            \mycomma
        \end{equation*}
        essendo naturalmente $\dd{T_n} \leq 0$. Il calcolo di $\rho\,\abs{{\left\langle - \dd{E} / \dd{x} \right\rangle}_n}$ è svolto dalla funzione \texttt{bethe()} chiamata alla riga \lnref{ln:bet:bethe-call} e definita alla riga \lnref{ln:bet:bethe-def}, con l'ausilio della funzione \texttt{wmax()} che in particolare calcola il termine $W_\text{max}$.

        Il calcolo viene quindi ripetuto a partire dalla nuova energia $T_{n + 1}$ per ottenere la perdita di energia attraverso il successivo strato. Per ogni reiterazione il codice stampa su un file la distanza totale percorsa, l'energia cinetica della particella e la quantità $\abs{\dd{T_n}} = \rho\,\abs{{\left\langle - \dd{E} / \dd{x} \right\rangle}_n}$.

    \subsection{Risultati della simulazione}
        \begin{figure}
            \centering
            \begin{tikzpicture}
    \begin{axis}[
        width   = \plotsize,
        height  = 0.6\plotsize,
        xlabel  = {Distanza (\unit{\centi\meter})},
        ylabel  = {Energia $\alpha$ (\unit{\mega\eV})},
        xmin    = 0.0000,   xmax = 0.0022,  xtick = {0.0000,0.0002,...,0.0022},
        ymin    = 0.0,      ymax = 5.0,     ytick = {0.0,1.0,...,5.0},
        tick align = outside,
    ]
        \addplot+ [
            only marks,
            mark        = *,
            mark size   = 0.5pt,
            draw        = black,
            fill        = black,
        ] table [
            x index = 0,
            y index = 1,
            col sep = space
        ] {data/bethebloch/better-data.dat};
    \end{axis}
\end{tikzpicture}
            \begin{tikzpicture}
    \begin{axis}[
        width   = \plotsize,
        height  = 0.6\plotsize,
        xlabel  = {Distanza (\unit{\centi\meter})},
        ylabel  = {$\rho\left\langle\dd{T} / \dd{x} \right\rangle$  (\unit{\mega\eV\,\centi\meter^{-1}})},
        xmin    = 0.0000,   xmax = 0.0022,  xtick = {0.0000,0.0002,...,0.0022},
        ymin    = 0.00,     ymax = 4000,    ytick = {0.0,500,...,4000},
        tick align = outside,
    ]
        \addplot+ [
            only marks,
            mark        = *,
            mark size   = 0.5pt,
            draw        = black,
            fill        = black,
        ] table [
            x index = 0,
            y index = 2,
            col sep = space
        ] {data/bethebloch/better-data.dat};
    \end{axis}
\end{tikzpicture}
            \caption{In alto l'energia della particella $\alpha$ che diminuisce man mano che la particella penetra l'alluminio. In basso il potere d'arresto con l'evidente picco di Bragg.}
            \label{fig:bet:alpha}
        \end{figure}
        Osserviamo adesso i dati che si ottengono inserendo nel programma un numero arbitrariamente alto di step pari a \num{500}. In \figref{fig:bet:alpha} sono riportati in due grafici i valori dell'energia cinetica della particella $\alpha$ e del potere d'arresto del materiale al variare della distanza percorsa.
        
        Dal primo grafico risulta evidente la decrescita dell'energia della particella. Si nota che l'energia cinetica non si annulla del tutto, nonostante si avvicini ragionevolmente a \SI{0}{\mega\eV}: questo può essere dovuto a imprecisioni nel codice come il modo in cui vengono gestiti valori di $T$ negativi e valori di $\dd{T}$ che farebbero aumentare l'energia.

        Nel secondo grafico invece è riportato l'andamento del potere d'arresto, detto comunemente \emph{curva di Bragg}. L'energia depositata dalla particella $\alpha$ è inversamente proporzionale al quadrato della velocità, per questo subito prima del totale arresto si osserva il massimo deposito di energia nel tratto di grafico detto \emph{picco di Bragg}.