\chapter{Sommario}
    \lettrine[loversize=0.08, lines=2]{I}{n questo} documento sono raccolte le quattro relazioni brevi da svolgere durante il corso annuale di \emph{Laboratorio di Fisica 3} del Corso di Laurea in \emph{Fisica} presso l'Università degli Studi di Catania.

    Le esperienze sono esposte nei quattro capitoli seguenti:
    \begin{enumerate}
        \item \emph{Implementazione numerica della formula di Bethe--Bloch}. Attraverso un codice in C che implementa numericamente la formula di Bethe--Bloch ho simulato il passaggio di una particella $\alpha$ a \SI{5}{\mega\eV} attraverso un sottile foglio di alluminio, realizzando un grafico che rappresenta l'energia della particella e la quantità di energia ceduta in funzione della distanza percorsa dentro il materiale.
        \item \emph{Misura di temperature con Arduino}. Attraverso l'uso di un microcontrollore Arduino, un sensore di temperatura e un semplice codice ho misurato la variazione di temperatura di una stanza in seguito all'accensione del riscaldamento. Nella relazione analizzo qualitativamente i dati raccolti ed estrapolo una possibile funzione che ne modelli l'andamento.
        \item \emph{Misura di resistenze con un multimetro digitale}.
        \item \emph{Accettanza geometrica di un rivelatore}.
    \end{enumerate} 