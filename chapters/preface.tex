\chapter{Sommario}
    \lettrine[loversize=0.08, lines=2]{I}{n questo} documento sono raccolte le quattro relazioni brevi da svolgere durante il corso annuale di \emph{Laboratorio di Fisica 3} del Corso di Laurea in \emph{Fisica} presso l'Università degli Studi di Catania.

    Le esperienze sono esposte nei quattro capitoli seguenti:
    \begin{enumerate}
        \item \emph{Implementazione numerica della formula di Bethe--Bloch}. Attraverso un codice in C che implementa numericamente la formula di Bethe--Bloch ho simulato il passaggio di una particella $\alpha$ a \SI{5}{\mega\eV} attraverso un sottile foglio di alluminio, realizzando un grafico che rappresenta l'energia della particella e la quantità di energia ceduta in funzione della distanza percorsa dentro il materiale.
        \item \emph{Misura di temperature con Arduino}. Attraverso l'uso di un microcontrollore Arduino, un sensore di temperatura e un semplice codice ho misurato la variazione di temperatura di una stanza in seguito all'accensione del riscaldamento. Nella relazione analizzo qualitativamente i dati raccolti ed estrapolo una possibile funzione che ne modelli l'andamento.
        \item \emph{Misura di resistenze con un multimetro digitale}. Utilizzando un multimetro digitale ho effettuato la misura dei resistori forniti dal kit del multimetro, verificandone la distribuzione statistica. A partire dai risultati di questo studio ho confrontato le misure di alcune delle resistenze collegate in parallelo con i valori previsti teoricamente. Infine ho trovato la resistività di un anello d'argento sfruttando di nuovo una misura di resistenza e considerazioni geometriche.
        \item \emph{Accettanza geometrica di un rivelatore}. Con un altro codice in C ho applicato il metodo Montecarlo per valutare numericamente l'accettanza geometrica di un rivelatore a forma di dischetto in presenza di una sorgente di radiazioni isotropa ed estesa. Dai dati simulati ho realizzato delle immagini rappresentative del sistema e un istogramma che mostri la distribuzione della radiazione sul sensore.
    \end{enumerate}
    
    Riporto inoltre i collegamenti alle mie due repository su GitHub dove è possibile consultare e scaricare i codici sorgente in C qualora si desiderasse utilizzarli.
    \begin{enumerate}
        \item[\ref{ch:bet}.] \url{https://github.com/ImAure/bethe-bloch-simulation}
        \item[\ref{ch:acc}.] \url{https://github.com/ImAure/geometric-acceptance}
    \end{enumerate}
    Il codice realizzato per l'esperienza con Arduino, essendo molto più breve, è invece esposto in appendice \Sref{app:ard}.