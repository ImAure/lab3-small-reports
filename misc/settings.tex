%=================%
%   PAGE LAYOUT   %
%=================%

\settypeblocksize{*}{1.2\lxvchars}{*}   % lxvchars è una dimensione raccomandata che dipende dal font, utile per la leggibilità
\setlrmargins{*}{*}{1}                  % setta i margini in modo che siano 1:1 adeguandosi al typeblock
\setulmarginsandblock{1.5in}{*}{1.4}    % setta i margini superiore e inferiore {superiore}{inferiore}{rapporto}
\checkandfixthelayout                   % does the magic



%=========================%
%   HEADER & PAGESTYLES   %
%=========================%

\setlength{\headwidth}{\textwidth}

\makepagestyle{thesis}                                                     % definisce il nome dello stile
\makerunningwidth{thesis}{\headwidth}                                      % definisce la lunghezza dell'header
\makeheadrule{thesis}{\headwidth}{\normalrulethickness}                    % mette la riga nell'header
\makeheadposition{thesis}{flushright}{flushleft}{flushright}{flushleft}    % definisce la posizione dell'header
\makepsmarks{thesis}{%                                                     % da qui in poi non lo so
    \nouppercaseheads
    \createmark{chapter}{both}{shownumber}{\chaptername\ }{.\ } 
    \createmark{section}{right}{shownumber}{}{.\ }              
    \createplainmark{toc}{both}{\contentsname}                  
    \createplainmark{lof}{both}{\listfigurename}                
    \createplainmark{lot}{both}{\listtablename}
    \createplainmark{bib}{both}{\bibname}
    \createplainmark{index}{both}{\indexname}
    \createplainmark{glossary}{both}{\glossaryname}    
}

%% se oneside:
    \makeoddhead{thesis}{\bfseries\leftmark}{}{\bfseries\thepage}
%% se twoside:
    % \makeevenhead{thesis}{\bfseries\thepage}{}{\bfseries\leftmark}
    % \makeoddhead{thesis}{\bfseries\rightmark}{}{\bfseries

\chapterstyle{hangnum}
\aliaspagestyle{chapter}{empty}                                             % toglie il numero di pagina nelle pagine con il nome del capitolo.
\addto\captionsitalian{\renewcommand{\chaptername}{Esperimento}}


%==============%
%   INITIALS   %
%==============%

\input Rothdn.fd
\newcommand*\initfamily{\usefont{U}{Rothdn}{xl}{n}}
\DeclareFontFamily{U}{yinit}{}
\DeclareFontShape{U}{yinit}{m}{n}{<-> yinit}{}
\newcommand{\initcolor}{purple}



%==============%
%   LISTINGS   %
%==============%

\definecolor{codegreen}{rgb}{0.4,0.6,0.4}
\definecolor{codegray}{rgb}{0.5,0.5,0.5}
\definecolor{codepurple}{rgb}{0.58,0.20,0.82}
\definecolor{codeorange}{rgb}{0.8,0.4,0.3}
\definecolor{backcolour}{rgb}{0.97,0.97,0.97}

\lstdefinestyle{mystyle}{
    backgroundcolor     = \color{backcolour},   
    commentstyle        = \color{codegreen},
    keywordstyle        = \color{codeorange},
    keywordstyle        = [2]\color{codeorange},
    %morekeyowrds        = [2]{SENSOR_PIN,MAX_READ,MAX_V,N,A,B,setup,pinMode,Serial,begin,loop},
    numberstyle         = \tiny\color{codegray},
    stringstyle         = \color{codepurple},
    basicstyle          = \ttfamily\footnotesize,
    breakatwhitespace   = true,         
    breaklines          = true,
    breakindent         = 0pt,                 
    captionpos          = b,                    
    keepspaces          = true,                 
    numbers             = left,
    frame               = tblr,
    rulecolor           = \color{black},
    numbersep           = 8pt,                  
    showspaces          = false,                
    showstringspaces    = false,
    showtabs            = false,                  
    tabsize             = 4,
    columns             = fixed,
}
\lstset{style=mystyle}



%===============%
%   NUMBERING   %
%===============%

\numberwithin{equation}{chapter}    % Aggiunge il numero del capitolo all'equazione
\setsecnumdepth{subsection}         % numera le subsection (1.1.1)

\hypersetup{                    % ======================================
    colorlinks=true,            % template preso da internet
    linkcolor=black,            % molto sobrio, le cose cliccabili si evidenziano quando passi il mouse
    filecolor=black,            %
    urlcolor=blue,              %
    pdfpagemode=FullScreen,     %
    citecolor=black,            %
}



%===========%
%   PLOTS   %
%===========%

\pgfplotsset{compat = 1.18}%, every axis/.append style={unbounded coords=jump, fpu=true}}
%\usepgfplotslibrary{external}
%\tikzexternalize 
\newlength{\plotsize}
\setlength{\plotsize}{0.8\textwidth}



%==================%
%   BIBLIOGRAPHY   %
%==================%

\addbibresource{misc/references.bib}