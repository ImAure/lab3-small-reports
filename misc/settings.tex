%=================%
%   PAGE LAYOUT   %
%=================%

\settypeblocksize{*}{1.2\lxvchars}{*}   % lxvchars è una dimensione raccomandata che dipende dal font, utile per la leggibilità
\setlrmargins{*}{*}{1}                  % setta i margini in modo che siano 1:1 adeguandosi al typeblock
\setulmarginsandblock{1.5in}{*}{1.4}    % setta i margini superiore e inferiore {superiore}{inferiore}{rapporto}
\checkandfixthelayout                   % does the magic



%=========================%
%   HEADER & PAGESTYLES   %
%=========================%

\setlength{\headwidth}{\textwidth}

\makepagestyle{thesis}                                                     % definisce il nome dello stile
\makerunningwidth{thesis}{\headwidth}                                      % definisce la lunghezza dell'header
\makeheadrule{thesis}{\headwidth}{\normalrulethickness}                    % mette la riga nell'header
\makeheadposition{thesis}{flushright}{flushleft}{flushright}{flushleft}    % definisce la posizione dell'header
\makepsmarks{thesis}{%                                                     % da qui in poi non lo so
    \nouppercaseheads
    \createmark{chapter}{both}{shownumber}{\chaptername\ }{.\ } 
    \createmark{section}{right}{shownumber}{}{.\ }              
    \createplainmark{toc}{both}{\contentsname}                  
    \createplainmark{lof}{both}{\listfigurename}                
    \createplainmark{lot}{both}{\listtablename}
    \createplainmark{bib}{both}{\bibname}
    \createplainmark{index}{both}{\indexname}
    \createplainmark{glossary}{both}{\glossaryname}    
}

%% se oneside:
    \makeoddhead{thesis}{\bfseries\leftmark}{}{\bfseries\thepage}
%% se twoside:
    % \makeevenhead{thesis}{\bfseries\thepage}{}{\bfseries\leftmark}
    % \makeoddhead{thesis}{\bfseries\rightmark}{}{\bfseries}

\chapterstyle{hangnum}
\aliaspagestyle{chapter}{empty}                                             % toglie il numero di pagina nelle pagine con il nome del capitolo.
\addto\captionsitalian{\renewcommand{\chaptername}{Esperimento}}


%==============%
%   INITIALS   %
%==============%

\input Rothdn.fd
\newcommand*\initfamily{\usefont{U}{Rothdn}{xl}{n}}
\DeclareFontFamily{U}{yinit}{}
\DeclareFontShape{U}{yinit}{m}{n}{<-> yinit}{}
\newcommand{\initcolor}{purple}



%==============%
%   LISTINGS   %
%==============%

\definecolor{codegreen}{rgb}{0.4,0.6,0.4}
\definecolor{codegray}{rgb}{0.5,0.5,0.5}
\definecolor{codepurple}{rgb}{0.58,0.20,0.82}
\definecolor{codeorange}{rgb}{0.8,0.4,0.3}
\definecolor{backcolour}{rgb}{0.97,0.97,0.97}

\lstloadlanguages{C, C++, bash}

\definecolor{mGreen}{rgb}{0,0.5,0}
\definecolor{mGray}{rgb}{0.5,0.5,0.5}
\definecolor{mPurple}{rgb}{0.6,0.2,0.7}
\definecolor{mBlue}{rgb}{0.2,0.2,0.9}
\definecolor{mLightBlue}{rgb}{0,0.6,0.6}
\definecolor{mOrange}{rgb}{0.8,0.3,0.2}
\definecolor{backgroundColour}{rgb}{0.95,0.95,0.92}

\lstdefinestyle{CStyle}{
    language          = C,
    backgroundcolor   = \color{backgroundColour},   
    commentstyle      = \color{mGray},
    otherkeywords     = {+, =, -, *, &, /, ?, :, <, >, ;},
    keywordstyle      =    \color{mPurple},
    keywordstyle      = [2]\color{mBlue},
    keywordstyle      = [3]\color{mOrange},
    keywordstyle      = [4]\color{mLightBlue},
    morekeywords      =    {polar2D_t, polar3D_t, file_t, point2D_t, point3D_t, disc3D_t},
    morekeywords      = [2]{rand_polar2D, sqrt, rand, rand_polar3D, cbrt, acos, sin, cos, tan, cart_to_polar2D, fprintf, intercept, wmax, bethe, log},
    morekeywords      = [3]{NULL, RAND_MAX, M_PI, E_M},
    morekeywords      = [4]{+, =, -, *, &, /, ?, :, <, >, ;},
    numberstyle       = \tiny\color{mGray},
    stringstyle       = \color{mGreen},
    basicstyle        = \ttfamily\footnotesize,
    breakatwhitespace = false,         
    breaklines        = true,   
    breakindent       = 0pt,              
    captionpos        = b,                    
    keepspaces        = true,                 
    numbers           = left,    
    frame             = tblr,
    rulecolor         = \color{black},
    numbersep         = 8pt,                                  
    showspaces        = false,                
    showstringspaces  = false,
    showtabs          = false,       
    columns           = fixed,           
    tabsize           = 4
}

\lstdefinestyle{myarduino}{
    language          = C++,
    backgroundcolor   = \color{backgroundColour},   
    commentstyle      = \color{mGray},
    keywordstyle      = \color{mPurple},
    otherkeywords     = {+, =, -, *, /, ?, :, <, >, ;},
    keywordstyle      = [2]\color{mOrange},
    keywordstyle      = [3]\color{mBlue},
    keywordstyle      = [4]\color{mLightBlue},
    morekeywords      = [2]{SENSOR_PIN, INPUT, MAX_READ, MAX_V, N, A, B, 9600},
    morekeywords      = [3]{setup, loop, print, analogRead, println, millis, delay, pinMode, begin},
    morekeywords      = [4]{+, =, -, *, /, ?, :, <, >, ;},
    morecomment       = [l]{//},
    numberstyle       = \tiny\color{mGray},
    stringstyle       = \color{mGreen},
    basicstyle        = \ttfamily\footnotesize,
    breakatwhitespace = true,         
    breaklines        = true,
    breakindent       = 0pt,                 
    captionpos        = b,                    
    keepspaces        = true,                 
    numbers           = left,
    frame             = tblr,
    rulecolor         = \color{black},
    numbersep         = 8pt,                  
    showspaces        = false,                
    showstringspaces  = false,
    showtabs          = false,                  
    tabsize           = 4,
    columns           = fixed,
}

\lstdefinestyle{mybash}{
    language          = bash,
    backgroundcolor   = \color{backgroundColour},   
    commentstyle      = \color{mGray},
    numberstyle       = \tiny\color{mGray},
    stringstyle       = \color{mGreen},
    alsoletter        = {., /, \$},
    keywordstyle      =    \color{mBlue},
    keywordstyle      = [2]\color{mGreen},
    morekeywords      =    {./sim},
    morekeywords      = [2]{\$},
    basicstyle        = \ttfamily\footnotesize,
    breakatwhitespace = true,         
    breaklines        = true,
    breakindent       = 0pt,                 
    captionpos        = b,                    
    keepspaces        = true,                 
    frame             = tblr,
    rulecolor         = \color{black},
    showspaces        = false,                
    showstringspaces  = false,
    showtabs          = false,                  
    tabsize           = 4,
    columns           = fixed,
}



%===============%
%   NUMBERING   %
%===============%

\numberwithin{equation}{chapter}    % Aggiunge il numero del capitolo all'equazione
\setsecnumdepth{subsection}         % numera le subsection (1.1.1)

\hypersetup{                    % ======================================
    colorlinks=true,            % template preso da internet
    linkcolor=black,            % molto sobrio, le cose cliccabili si evidenziano quando passi il mouse
    filecolor=black,            %
    urlcolor=mBlue,              %
    pdfpagemode=FullScreen,     %
    citecolor=black,            %
}



%===========%
%   PLOTS   %
%===========%

\pgfplotsset{compat = 1.18}%, every axis/.append style={unbounded coords=jump, fpu=true}}
%\usepgfplotslibrary{external}
%\tikzexternalize 
\newlength{\plotsize}
\setlength{\plotsize}{0.8\textwidth}



%==========================%
%   DON'T ASK ABOUT THIS   %
%==========================%

\ExplSyntaxOn
    \msg_redirect_name:nnn{siunitx}{physics-pkg}{none} % shut up the warning. \qty{} uses physics definition
\ExplSyntaxOff



%==================%
%   BIBLIOGRAPHY   %
%==================%

\addbibresource{misc/references.bib}