\documentclass[oneside, a4paper, 11pt, final]{memoir}
\usepackage[T1]{fontenc}                % per il font
\usepackage[utf8]{inputenc}             % per l'input
\usepackage[english, italian]{babel}    % lingua principale inglese, secondaria italiano
%\input Rothdn.fd
%\newcommand*\initfamily{\usefont{U}{Rothdn}{xl}{n}}

\usepackage{layout}                 % per testare la formattazione del testo
\usepackage{lipsum}                 % anche lui per creare testo a caso
\usepackage{geometry}               % serve

\usepackage[intlimits]{amsmath}     % =======================================
\usepackage{amsfonts}               % cose utili per fare matematica e fisica
\usepackage{amssymb}                % intlimits mette gli estremi di integrazione sopra e sotto il segno di integrale       
\usepackage{amsthm}                 % cose utili per fare matematica e fisica
\usepackage{bbm}                    % cose utili per fare matematica e fisica
\usepackage{physics}                % cose utili per fare matematica e fisica
\usepackage{tensor}                 % =======================================
\usepackage{siunitx}
%\usepackage[scaled]{beramono}
\usepackage{inconsolata}

\usepackage{graphicx}               % per le foto
\usepackage{caption}                % per le caption
\usepackage{tikz}                   % per fare i disegnini belli
\usepackage{pgf}
\usepackage{pgfplots}
\usepackage{enumitem}               % 
\usepackage{listings}
\usepackage{emptypage}              % non se se serve
\usepackage{lettrine}               % per le lettere belle a inizio capitolo
%\usepackage{yfonts}                 % per le lettere belle a inizio capitolo
\input Rothdn.fd
\newcommand*\initfamily{\usefont{U}{Rothdn}{xl}{n}}
\DeclareFontFamily{U}{yinit}{}
\DeclareFontShape{U}{yinit}{m}{n}{<-> yinit}{}
\usepackage{hyperref}               % fa le cross reference cliccabili dentro al testo (ad esempio da indice a capitolo)
                                    % ======================================

\usepackage{xurl}
\usepackage{csquotes}
\usepackage[backend = bibtex, style = numeric-comp]{biblatex}
    \addbibresource{misc/references.bib}

\usepackage{misc/thesisutils}
%=================%
%   PAGE LAYOUT   %
%=================%

\settypeblocksize{*}{1.2\lxvchars}{*}   % lxvchars è una dimensione raccomandata che dipende dal font, utile per la leggibilità
\setlrmargins{*}{*}{1}                  % setta i margini in modo che siano 1:1 adeguandosi al typeblock
\setulmarginsandblock{1.5in}{*}{1.4}    % setta i margini superiore e inferiore {superiore}{inferiore}{rapporto}
\checkandfixthelayout                   % does the magic



%=========================%
%   HEADER & PAGESTYLES   %
%=========================%

\setlength{\headwidth}{\textwidth}

\makepagestyle{thesis}                                                     % definisce il nome dello stile
\makerunningwidth{thesis}{\headwidth}                                      % definisce la lunghezza dell'header
\makeheadrule{thesis}{\headwidth}{\normalrulethickness}                    % mette la riga nell'header
\makeheadposition{thesis}{flushright}{flushleft}{flushright}{flushleft}    % definisce la posizione dell'header
\makepsmarks{thesis}{%                                                     % da qui in poi non lo so
    \nouppercaseheads
    \createmark{chapter}{both}{shownumber}{\chaptername\ }{.\ } 
    \createmark{section}{right}{shownumber}{}{.\ }              
    \createplainmark{toc}{both}{\contentsname}                  
    \createplainmark{lof}{both}{\listfigurename}                
    \createplainmark{lot}{both}{\listtablename}
    \createplainmark{bib}{both}{\bibname}
    \createplainmark{index}{both}{\indexname}
    \createplainmark{glossary}{both}{\glossaryname}    
}

%% se oneside:
    \makeoddhead{thesis}{\bfseries\leftmark}{}{\bfseries\thepage}
%% se twoside:
    % \makeevenhead{thesis}{\bfseries\thepage}{}{\bfseries\leftmark}
    % \makeoddhead{thesis}{\bfseries\rightmark}{}{\bfseries}

\chapterstyle{hangnum}
\aliaspagestyle{chapter}{empty}                                             % toglie il numero di pagina nelle pagine con il nome del capitolo.
\addto\captionsitalian{\renewcommand{\chaptername}{Esperimento}}


%==============%
%   INITIALS   %
%==============%

\input Rothdn.fd
\newcommand*\initfamily{\usefont{U}{Rothdn}{xl}{n}}
\DeclareFontFamily{U}{yinit}{}
\DeclareFontShape{U}{yinit}{m}{n}{<-> yinit}{}
\newcommand{\initcolor}{purple}



%==============%
%   LISTINGS   %
%==============%

\definecolor{codegreen}{rgb}{0.4,0.6,0.4}
\definecolor{codegray}{rgb}{0.5,0.5,0.5}
\definecolor{codepurple}{rgb}{0.58,0.20,0.82}
\definecolor{codeorange}{rgb}{0.8,0.4,0.3}
\definecolor{backcolour}{rgb}{0.97,0.97,0.97}

\lstloadlanguages{C, C++, bash}

\definecolor{mGreen}{rgb}{0,0.5,0}
\definecolor{mGray}{rgb}{0.5,0.5,0.5}
\definecolor{mPurple}{rgb}{0.6,0.2,0.7}
\definecolor{mBlue}{rgb}{0.2,0.2,0.9}
\definecolor{mLightBlue}{rgb}{0,0.6,0.6}
\definecolor{mOrange}{rgb}{0.8,0.3,0.2}
\definecolor{backgroundColour}{rgb}{0.95,0.95,0.92}

\lstdefinestyle{CStyle}{
    language          = C,
    backgroundcolor   = \color{backgroundColour},   
    commentstyle      = \color{mGray},
    otherkeywords     = {+, =, -, *, &, /, ?, :, <, >, ;},
    keywordstyle      =    \color{mPurple},
    keywordstyle      = [2]\color{mBlue},
    keywordstyle      = [3]\color{mOrange},
    keywordstyle      = [4]\color{mLightBlue},
    morekeywords      =    {polar2D_t, polar3D_t, file_t, point2D_t, point3D_t, disc3D_t},
    morekeywords      = [2]{rand_polar2D, sqrt, rand, rand_polar3D, cbrt, acos, sin, cos, tan, cart_to_polar2D, fprintf, intercept, wmax, bethe, log},
    morekeywords      = [3]{NULL, RAND_MAX, M_PI, E_M},
    morekeywords      = [4]{+, =, -, *, &, /, ?, :, <, >, ;},
    numberstyle       = \tiny\color{mGray},
    stringstyle       = \color{mGreen},
    basicstyle        = \ttfamily\footnotesize,
    breakatwhitespace = false,         
    breaklines        = true,   
    breakindent       = 0pt,              
    captionpos        = b,                    
    keepspaces        = true,                 
    numbers           = left,    
    frame             = tblr,
    rulecolor         = \color{black},
    numbersep         = 8pt,                                  
    showspaces        = false,                
    showstringspaces  = false,
    showtabs          = false,       
    columns           = fixed,           
    tabsize           = 4
}

\lstdefinestyle{myarduino}{
    language          = C++,
    backgroundcolor   = \color{backgroundColour},   
    commentstyle      = \color{mGray},
    keywordstyle      = \color{mPurple},
    otherkeywords     = {+, =, -, *, /, ?, :, <, >, ;},
    keywordstyle      = [2]\color{mOrange},
    keywordstyle      = [3]\color{mBlue},
    keywordstyle      = [4]\color{mLightBlue},
    morekeywords      = [2]{SENSOR_PIN, INPUT, MAX_READ, MAX_V, N, A, B, 9600},
    morekeywords      = [3]{setup, loop, print, analogRead, println, millis, delay, pinMode, begin},
    morekeywords      = [4]{+, =, -, *, /, ?, :, <, >, ;},
    morecomment       = [l]{//},
    numberstyle       = \tiny\color{mGray},
    stringstyle       = \color{mGreen},
    basicstyle        = \ttfamily\footnotesize,
    breakatwhitespace = true,         
    breaklines        = true,
    breakindent       = 0pt,                 
    captionpos        = b,                    
    keepspaces        = true,                 
    numbers           = left,
    frame             = tblr,
    rulecolor         = \color{black},
    numbersep         = 8pt,                  
    showspaces        = false,                
    showstringspaces  = false,
    showtabs          = false,                  
    tabsize           = 4,
    columns           = fixed,
}

\lstdefinestyle{mybash}{
    language          = bash,
    backgroundcolor   = \color{backgroundColour},   
    commentstyle      = \color{mGray},
    numberstyle       = \tiny\color{mGray},
    stringstyle       = \color{mGreen},
    alsoletter        = {., /, \$},
    keywordstyle      =    \color{mBlue},
    keywordstyle      = [2]\color{mGreen},
    morekeywords      =    {./sim},
    morekeywords      = [2]{\$},
    basicstyle        = \ttfamily\footnotesize,
    breakatwhitespace = true,         
    breaklines        = true,
    breakindent       = 0pt,                 
    captionpos        = b,                    
    keepspaces        = true,                 
    frame             = tblr,
    rulecolor         = \color{black},
    showspaces        = false,                
    showstringspaces  = false,
    showtabs          = false,                  
    tabsize           = 4,
    columns           = fixed,
}



%===============%
%   NUMBERING   %
%===============%

\numberwithin{equation}{chapter}    % Aggiunge il numero del capitolo all'equazione
\setsecnumdepth{subsection}         % numera le subsection (1.1.1)

\hypersetup{                    % ======================================
    colorlinks=true,            % template preso da internet
    linkcolor=black,            % molto sobrio, le cose cliccabili si evidenziano quando passi il mouse
    filecolor=black,            %
    urlcolor=mBlue,              %
    pdfpagemode=FullScreen,     %
    citecolor=black,            %
}



%===========%
%   PLOTS   %
%===========%

\pgfplotsset{compat = 1.18}%, every axis/.append style={unbounded coords=jump, fpu=true}}
%\usepgfplotslibrary{external}
%\tikzexternalize 
\newlength{\plotsize}
\setlength{\plotsize}{0.8\textwidth}



%==========================%
%   DON'T ASK ABOUT THIS   %
%==========================%

\ExplSyntaxOn
    \msg_redirect_name:nnn{siunitx}{physics-pkg}{none} % shut up the warning. \qty{} uses physics definition
\ExplSyntaxOff



%==================%
%   BIBLIOGRAPHY   %
%==================%

\addbibresource{misc/references.bib}
\DeclareFontFamily{U}{yinit}{}
\DeclareFontShape{U}{yinit}{m}{n}{<-> yinit}{}
% ---- REIMPOSTAZIONI ----
%\font\nullfont=cmr10
\begin{document}
    %\newgeometry{margin=50pt}
\begin{titlingpage}
    \begin{center}
        centro
    \end{center}
\end{titlingpage}
\restoregeometry
    \pagestyle{thesis}

    \frontmatter
        %\tableofcontents
        %\chapter{Sommario}
    \lettrine[loversize=0.08, lines=2]{I}{n questo} documento sono raccolte le quattro relazioni brevi da svolgere durante il corso annuale di \emph{Laboratorio di Fisica 3} del Corso di Laurea in \emph{Fisica} presso l'Università degli Studi di Catania.

    Le esperienze sono esposte nei quattro capitoli seguenti:
    \begin{enumerate}
        \item \emph{Implementazione numerica della formula di Bethe--Bloch}. Attraverso un codice in C che implementa numericamente la formula di Bethe--Bloch ho simulato il passaggio di una particella $\alpha$ a \SI{5}{\mega\eV} attraverso un sottile foglio di alluminio, realizzando un grafico che rappresenta l'energia della particella e la quantità di energia ceduta in funzione della distanza percorsa dentro il materiale.
        \item \emph{Misura di temperature con Arduino}. Attraverso l'uso di un microcontrollore Arduino, un sensore di temperatura e un semplice codice ho misurato la variazione di temperatura di una stanza in seguito all'accensione del riscaldamento. Nella relazione analizzo qualitativamente i dati raccolti ed estrapolo una possibile funzione che ne modelli l'andamento.
        \item \emph{Misura di resistenze con un multimetro digitale}. Utilizzando un multimetro digitale ho effettuato la misura dei resistori forniti dal kit del multimetro, verificandone la distribuzione statistica. A partire dai risultati di questo studio ho confrontato le misure di alcune delle resistenze collegate in parallelo con i valori previsti teoricamente. Infine ho trovato la resistività di un anello d'argento sfruttando di nuovo una misura di resistenza e considerazioni geometriche.
        \item \emph{Accettanza geometrica di un rivelatore}. Con un altro codice in C ho applicato il metodo Montecarlo per valutare numericamente l'accettanza geometrica di un rivelatore a forma di dischetto in presenza di una sorgente di radiazioni isotropa ed estesa. Dai dati simulati ho realizzato delle immagini rappresentative del sistema e un istogramma che mostri la distribuzione della radiazione sul sensore.
    \end{enumerate}
    
    Riporto inoltre i collegamenti alle mie due repository su GitHub dove è possibile consultare e scaricare i codici sorgente in C qualora si desiderasse utilizzarli.
    \begin{enumerate}
        \item[\ref{ch:bet}.] \url{https://github.com/ImAure/bethe-bloch-simulation}
        \item[\ref{ch:acc}.] \url{https://github.com/ImAure/geometric-acceptance}
    \end{enumerate}

    Le parti di codice necessarie alla descrizione dei progetti saranno di volta in volta richiamate ove necessario. Il codice realizzato per l'esperienza con Arduino, essendo sufficientemente breve, è invece interamente esposto in appendice \Sref{app:ard}.

    \bigskip\noindent
    Catania\hfill\textsc{A. Pappalardo}\\\today

    \mainmatter
        \chapter[Implementazione numerica della Formula di Bethe--Bloch][Formula di Bethe--Bloch]{Implementazione Numerica della Formula di Bethe--Bloch}\label{ch:bet}

\section{Il modello}
    \lettrine[loversize=0.08, lines=2]{C}{ome descritto} dal Particle Data Group \cite{PhysRevD-110-030001}, la formula di Bethe--Bloch è un modello sperimentale che descrive la perdita di energia di particelle cariche pesanti di media energia---come protoni e particelle $\alpha$---nella materia:
    \begin{equation}
        \left\langle -\dv{E}{x}\right\rangle
        = Kz^2 \frac{Z}{A}\frac{1}{\beta^2}\bqty{\frac{1}{2}\log\frac{2 m_e c^2 \beta^2 \gamma^2 W_\text{max}}{I^2} - \beta^2 - \frac{\delta\pqty{\beta\gamma}}{2}}
        \mycomma
        \label{eq:bet:bethe-bloch-1}
    \end{equation}
    dove $\beta$ e $\gamma$ sono le usuali quantità relativistiche mentre il resto dei simboli sono esplicitati in \tabref{tab:bet:costanti}.
    \begin{table}
        \footnotesize
        \centering
        \begin{tabular*}{\textwidth}{@{\extracolsep{\fill}}lll}\hline\rule{0pt}{8pt}%
    Simbolo & Definizione & Valore o unità di misura\\[0.5pt]
    \hline\hline\rule{0pt}{9pt}%
    $m_e c^2$ & massa a riposo dell'elettrone $\times c^2$ & \SI{ 0.51099895000(15)}{\mega\eV}\\
    $r_e$ & raggio classico dell'elettrone $ e^2/4\pi \epsilon_0 m_e c^2$ & \SI{2.8179403227(19)}{\femto\meter}\\
    $N_\text{A}$ & numero di Avogadro & \SI{ 6.02214076e+23}{mol^{-1}}\\[2pt]
    \hline\rule{0pt}{9pt}%
    $\rho$ & densità & \unit{\g\,\centi\meter^{-3}}\\
    $x$ & massa per unità di area & \unit{\gram\,\centi\meter^{-2}}\\ 
    $M$ & massa della particella incidente & \unit{\mega\eV \,\mathit{c}^{-2}}\\
    $E$ & energia della particella incidente $\gamma M c^2$ & \unit{\mega\eV}\\
    $W_\text{max}$ & massima energia trasferibile per collisioni & \unit{\mega\eV}\\
    $z$ & numero di carica della particella incidente & \\
    $Z$ & numreo atomico del bersaglio & \\
    $A$ & numero di massa atomica del bersaglio & \\
    $K$ & $4\pi N_\text{A} r_e^2 m_e c^2$ & \SI{0.307075}{\mega\eV\,\mol^{-1}\,\centi\meter^2}\\
    $I$ & energia media di eccitazione & \unit{\eV}\\
    $\delta\pqty{\beta\gamma}$ & correzione di ionizzazione & \\[1pt]
    \hline
\end{tabular*}
        \caption{Notazione e unità di misura per la formula di Bethe--Bloch. Si tratta di un riassunto della tabella del PDG \cite{PhysRevD-110-030001}.}
        \label{tab:bet:costanti}
    \end{table}
    La perdita di energia media data dalla \eqref{eq:bet:bethe-bloch-1} è misuarta in \unit{\mega\eV\,\gram^{-1}\,\centi\meter^2} ma può essere portata in \unit{\mega\eV\,\centi\meter^{-1}} moltiplicando entrambi i membri per la densità volumica $\rho$ del bersaglio misurata in \unit{\gram\,\centi\meter^{-3}}. La quantità $W_\text{max}$ è la massima energia che una particella carica può cedere a un elettrone e si esprime come
    \begin{equation}
        W_\text{max} = \frac{2 m_e c^2 \beta^2 \gamma^2}{1 + \gamma\, m_e / M + (m_e / M)^2}
        \myperiod 
        \label{eq:bet:wmax}
    \end{equation}
    La \eqref{eq:bet:bethe-bloch-1} e la \eqref{eq:bet:wmax} sono valide nell'approssimazione $\num{0.1} \lesssim \beta\gamma \lesssim \num{1000}$, poiché al limite inferiore la velocità del proiettile diventa confrontabile con la ``velocità'' degli elettroni atomici mentre al limite superiore gli effetti radiativi non sono più trascurabili.

\section{La simulazione}
    Ho scelto di simulare la perdita di energia di una particella $\alpha$ a \SI{5}{\mega\eV} attraverso un foglio di alluminio di spessore\footnote{Dopo aver provato con uno spessore di \SI{0.015}{\milli\meter} e aver constatato che la curva risultava tagliata---la particella non cedeva tutta l'energia all'alluminio---ho scelto di aumentare di poco lo spessore per il gusto di un grafico più completo.} \SI{0.022}{\milli\meter}. Per realizzare la simulazione ho scritto un codice in C che implementa la \eqref{eq:bet:bethe-bloch-1} in modo numerico.
    \subsection{Descrizione del codice e dei calcoli}
        L'utente che esegue il programma sceglie lo spessore del materiale bersaglio e il numero di step in cui suddividere il calcolo. Il programma quindi chiede all'utente di selezionare un proiettile, tra $\alpha$, protoni e muoni, e un materiale tra alluminio e rame. I dati come le masse delle particelle, le densità dei materiali e i valori di $I$, sono definiti come costanti all'inizio del programma---non li riporto in questo elaborato per brevità; la totalità del codice e delle costanti incluse è consultabile su GitHub.\footnote{Repository: \url{https://github.com/ImAure/bethe-bloch-simulation}}
        
        Assumiamo che in ciascun intervallo di spessore infinitesimo\footnote{Ricordo che in questo caso la $x$ non indica una distanza.} $\dd{s}$ tutte le quantità variabili di nostro interesse siano costanti---velocità, energia, perdita di energia \myetc. Dalla teoria della relatività ristretta scriviamo per l'$n$-esimo intervallo
        \begin{align*}   
                E_n &= T_n + m c^2 \mycomma \\
                E_n &= \gamma_n m c^2 \mycomma
        \end{align*}
        dove $m$ è la massa a riposo della particella incidente, $T_n$ la sua energia cinetica ed $E_n$ la sua energia totale. Note queste ultime due quantità, è possibile calcolare il fattore di Lorentz $\gamma_n$ e il valore $\beta_n^2$ come
        \begin{equation*}
            \gamma_n = \frac{T_n + m c^2}{m c^2}
            \,,
            \quad
            \beta_n^2 = 1 - \frac{1}{\gamma_n^2}
            \myperiod
        \end{equation*}

        Noti $\gamma_n$ e $\beta_n$, è possibile calcolare la perdita di energia media per unità di lunghezza attraverso la \eqref{eq:bet:bethe-bloch-1} moltiplicata per $\rho$. Possiamo in particolare calcolare l'energia cinetica con cui la particella $\alpha$ entra nello strato successivo, $T_{n + 1}$, come
        \begin{equation*}
            T_{n + 1} = T_n + \dd{T_n} = T_n + \rho\, {\left\langle - \dv{E}{x} \right\rangle}_n \dd{s}
            \mycomma
        \end{equation*}
        essendo naturalmente $\dd{T_n} \leq 0$.
        
        Il calcolo di $\rho\,\abs{{\left\langle - \dd{E} / \dd{x} \right\rangle}_n}$ è svolto dalla funzione \texttt{bethe()}, con l'ausilio della funzione \texttt{wmax()} che in particolare calcola il termine $W_\text{max}$. Nel fare questo conto viene trascurato il termine $\delta\pqty{\beta\gamma}/2$.
        \lstinputlisting[style = CStyle]{code/bethebloch/bethe-and-wmax.txt}
        Il procedimento viene quindi ripetuto a partire dalla nuova energia $T_{n + 1}$ per ottenere la perdita di energia attraverso il successivo strato. Per ogni reiterazione il codice stampa su un file la distanza totale percorsa, l'energia cinetica della particella e la quantità $\abs{\dd{T_n} / \dd{s}} = \rho\,\abs{{\left\langle - \dd{E} / \dd{x} \right\rangle}_n}$.

    \subsection{Risultati della simulazione}
        \begin{figure}
            \centering
            \begin{tikzpicture}
    \begin{axis}[
        width   = \plotsize,
        height  = 0.6\plotsize,
        xlabel  = {Distanza (\unit{\centi\meter})},
        ylabel  = {Energia $\alpha$ (\unit{\mega\eV})},
        xmin    = 0.0000,   xmax = 0.0022,  xtick = {0.0000,0.0002,...,0.0022},
        ymin    = 0.0,      ymax = 5.0,     ytick = {0.0,1.0,...,5.0},
        tick align = outside,
    ]
        \addplot+ [
            only marks,
            mark        = *,
            mark size   = 0.5pt,
            draw        = black,
            fill        = black,
        ] table [
            x index = 0,
            y index = 1,
            col sep = space
        ] {data/bethebloch/better-data.dat};
    \end{axis}
\end{tikzpicture}
            \caption{L'energia della particella $\alpha$ diminuisce man mano che la particella penetra nell'alluminio, non si annulla del tutto probabilmente per errori numerici.}
            \label{fig:bet:alpha-1}
        \end{figure}
        \begin{figure}
            \centering
            \begin{tikzpicture}
    \begin{axis}[
        width   = \plotsize,
        height  = 0.6\plotsize,
        xlabel  = {Distanza (\unit{\centi\meter})},
        ylabel  = {$\rho\left\langle\dd{T} / \dd{x} \right\rangle$  (\unit{\mega\eV\,\centi\meter^{-1}})},
        xmin    = 0.0000,   xmax = 0.0022,  xtick = {0.0000,0.0002,...,0.0022},
        ymin    = 0.00,     ymax = 4000,    ytick = {0.0,500,...,4000},
        tick align = outside,
    ]
        \addplot+ [
            only marks,
            mark        = *,
            mark size   = 0.5pt,
            draw        = black,
            fill        = black,
        ] table [
            x index = 0,
            y index = 2,
            col sep = space
        ] {data/bethebloch/better-data.dat};
    \end{axis}
\end{tikzpicture}
            \caption{Il potere d'arresto con l'evidente picco di Bragg.}
            \label{fig:bet:alpha-2}
        \end{figure}
        Osserviamo adesso i dati che si ottengono inserendo nel programma un numero arbitrariamente alto di step pari a \num{1000}. In \figref{fig:bet:alpha-1} e \figref{fig:bet:alpha-2} sono riportati in due grafici i valori dell'energia cinetica della particella $\alpha$ e del potere d'arresto del materiale al variare della distanza percorsa.
        
        Dal primo grafico risulta evidente la decrescita dell'energia della particella. Si nota che l'energia cinetica non si annulla del tutto, nonostante si avvicini ragionevolmente a \SI{0}{\mega\eV}: questo può essere dovuto a imprecisioni nel codice come il modo in cui vengono gestiti eventuali valori di $T < 0$ e valori di $\dd{T} > \num{0}$ che farebbero aumentare l'energia. In particolare, le formule implementate nel codice sono
        \begin{align*}
            \dd{T_n} &= \max\!\pqty{0,\,\rho\,{\left\langle - \dv{E}{x} \right\rangle}_n} \mycomma\\
            T_{n+1} &= \max\!\pqty{0,\,T_n + \dd{T_n}}\mycomma
        \end{align*}
        è quindi possibile che a un certo punto i $\dd{T_n}$ inizino a risultare positivi, vengano quindi imposti pari a \num{0} dal programma e la decrescita dell'energia si arresti di conseguenza. Un'altra possibile causa di questo comportamento è che il modello possa diventare inadatto a descrivere il comportamento di particelle a energie troppo basse, ovvero per $\beta\gamma \ll 0.1$.

        Nel secondo grafico invece è riportato l'andamento del potere d'arresto, detto comunemente \emph{curva di Bragg}. L'energia depositata dalla particella $\alpha$ dipende inversamente dal quadrato della velocità, per questo subito prima del totale arresto si osserva il massimo deposito di energia nel picco della curva, detto \emph{picco di Bragg}.
        \newcommand{\tmp}{TMP36}
\newcommand{\vs}{\texttt{+vs}}
\newcommand{\vout}{\texttt{vout}}
\newcommand{\gnd}{\texttt{gnd}}
\newcommand{\pinV}{\texttt{5v}}
\newcommand{\sensorpin}{\texttt{A0}}
\newcommand{\txtloop}{\texttt{loop}}

\chapter{Misura di Temperature con Arduino}
    Tra le esperienze svolte con Arduino Uno riporto in particolare la misura della variazione della temperatura della mia stanza da letto in seguito all'accensione del riscaldamento in casa.
    \section{L'esperimento}
            L'obiettivo dell'esperienza è quello di valutare qualitativamente l'andamento della temperatura della stanza per fare una stima di quanto velocemente si riscaldi e a quale temperatura tenda asintoticamente.
        \subsection{Preparazione della stanza}
            Per massimizzare l'escursione termica ho effettuato la misura durante una sera invernale avendo preventivamente aperto le finestre per abbassare la temperatura della stanza.
            
            Per migliorare la circolazione dell'aria ed evitare un eccessivo gradiente di temperatura---il radiatore caldo si trova in un angolo della stanza mentre i vetri freddi della finestra si trovano dal lato opposto---ho acceso dei ventilatori: uno a soffitto per limitare la raccolta dell'aria calda in alto e un più piccolo ventilatore da tavolo per allontanare l'aria calda dal radiatore e facilitare il riscaldamento dell'aria fredda.
            
            Infine, per isolare il più possbile il sistema, ho chiuso le tende sulla finestra per ridurre la dispersione di calore attraverso il vetro freddo e mantenuto la porta chiusa per non disperdere calore nel resto della casa.
        \subsection{Strumenti utilizzati}
            Gli strumenti utillizzati per la presa dei dati sono:
            \begin{enumerate}[label=$\bullet$]
                \item Una microcontrollore Arduino Uno con un sensore di temperatura \tmp;
                \item Un computer per compilare ed eseguire il codice sulla scheda Arduino e prelevare i dati.
            \end{enumerate}

            Il sensore \tmp\ è un sensore di temperatura a semiconduttore pensato per operare in un range di temperature che va da \SI{-40}{\celsius} a $+\SI{125}{\celsius}$. Esso presenta tre pin: \vs, \vout\ e \gnd. Il primo e l'ultimo servono per l'alimentazione che deve essere compresa tra \SI{2.7}{\volt} e \SI{5.5}{\volt} con una corrente inferiore ai \SI{50}{\micro\ampere}, che garantisce un surriscaldamento per effetto Joule trascurabile. Il secondo pin invece sestituisce una differenza di potenziale rispetto al \gnd\ proporzionale alla temperatura misurata. La sensibilità del sensore fornita dal costruttore è di \SI{\pm 1}{\celsius} e il suo fattore di scala è di \SI{10}{\milli\volt\per\celsius}  \cite{tmp36-datasheet}.

            La scheda Arduino attraverso i pin analogici accetta in input delle differenze di potenziale che vanno da \SI{0}{V} a \SI{5}{V} che vengono convertite in un segnale digitale che assume valori discreti da \num{0} a \num{1023}.
        \subsection{Circuito e codice}
            Il circuito realizzato per l'esperimento è quello rappresentato in \figref{fig:arduino-1}. L'alimentazione al sensore è fornita tramite i pin \pinV\ e \gnd\ mentre il segnale in uscita dal sensore viene letto dal pin \sensorpin\ della scheda.

            \begin{figure}[t]
                \centering
                    \begin{minipage}{0.49\textwidth}
                        \includegraphics[width = \textwidth]{images/arduino/temp-pic.png}
                    \end{minipage}
                    \hfill
                    \begin{minipage}{0.49\textwidth}
                        \includegraphics[width = \textwidth]{images/arduino/temp-scheme.pdf}
                    \end{minipage}
                    \caption{A sinistra una rappresentazione digitale del circuito realizzato per l'esperimento. A destra lo schema del circuito. Entrambe le illustrazioni sono state realizzate con Tinkercad\textsuperscript{\textregistered}.}
                    \label{fig:arduino-1}
            \end{figure}

            Per effettuare le misure ho usato il codice riportato di seguito. A intervalli di \SI{30}{\second} la lettura discreta di tensione data dal sensore\footnote{Come detto prima si tratta di un valore tra \num{0} e \num{1023}} e la converte in un numero decimale tra \SI{0}{\volt} e \SI{5}{V} attraverso la formula
            \begin{equation*}
                \frac{\text{\ttfamily (float)analogRead(SENSOR\_PIN)}}{\text{\ttfamily MAX\_READ}} * \text{\ttfamily MAX\_V}
                \mycomma
            \end{equation*}
            essendo $\text{\ttfamily MAX\_READ} = 1023$ e $\text{\ttfamily MAX\_V} = \SI{5}{\volt}$. Sapendo che a una tensione di \SI{0}{\volt} corrisponde una temperatura di \SI{0.5}{\celsius} e a \SI{4.5}{\volt} corrispondono \SI{100}{\celsius}, la conversione della lettura in gradi Celsius è data da
            \begin{equation}
                \bqty{\frac{\text{\ttfamily (float)analogRead(SENSOR\_PIN)}}{\text{\ttfamily MAX\_READ}} * \text{\ttfamily MAX\_V} - \text{\ttfamily A}} * \text{\ttfamily B}
                \mycomma
                \label{eq:arduino-temp-conv}
            \end{equation}
            dove $\text{\ttfamily A} = \SI{0.5}{\volt}$ e $\text{\ttfamily B} = \SI{100}{\celsius\per\volt}$ sono i fattori di scala.

            La conversione dei valori discreti in temperatura è eseguita dal codice tra le righe \texttt{23} e \texttt{26} applicando la \eqref{eq:arduino-temp-conv}. Vengono eseguite $\text{\ttfamily N} = 20$ misure consecuitive di cui è contestualmente calcolata la media che viene a sua volta stampata a schermo. Infine il codice attende il tempo mancante per raggiungere i \SI{30}{\second} dall'inizio del \txtloop.
            \lstinputlisting[language=C++]{code/arduino-temp.txt}
    \section{Dati}
        [aggiungere e commentare i dati]
        %\newcommand{\dvm}{DVM841}
\newcommand{\vell}{Vellemann\textsuperscript{\tiny\textregistered}}

\chapter{Misura di resistenze con un multimetro digitale}\label{ch:mult}
    \lettrine[loversize=0.08, lines=2]{T}{ra le esperienze} svolte con il multimetro digitale riporto la misura delle resistenze di vari oggetti, tra cui un anello e alcuni resistori, misurati sia individualmente che in parallelo.

    Ai resistori dedico una sezione più approfondita in quanto ho preso \num{50} misure su resistori distinti---ma teoricamente con resistenza uguale---per verificare la distribuzione delle misure.

    \section{Il multimetro}
        Lo strumento utilizzato per l'interezza dell'esperienza è un multimetro digitale della serie \dvm\ della \vell\ \cite{velleman-dvm841}, \figref{fig:mul:multimetro}. Il multimetro è in grado di misurare tensione e corrente continua e alternata, resistenza, frequenza e temperatura. Avendo una risoluzione di \num{2000} punti, il display del multimetro può visualizzare un massimo di \num{1999} unità.
        \begin{figure}
            \centering
            \includegraphics[width = 0.4\textwidth]{images/multimetro/multimetro.jpg}
            \caption{Il multimetro \dvm\ della \vell\ usato per l'esperienza.}
            \label{fig:mul:multimetro}
        \end{figure}

        L'apparecchio è alimentato a batteria e presenta tre prese a cui si possono collegare due puntali con gli appositi spinotti. Volendo misurare solo resistenze, ho usato solo la presa \unit{\volt\ohm\milli\ampere} e la messa a terra.

    \section{Resistori}\label{s:mul:resistori}
        Il kit presenta $N = \num{50}$ resistori distinti---come quelli in \figref{fig:mul:resistore}---il cui codice colore restituisce un valore\footnote{Lo si può dedurre da qualunque legenda fedele allo standard IEC 60062.} teorico di $\SI{820}{\ohm} \pm \SI{5}{\%}$, ovvero \SI{820(40)}{\ohm}.
        \begin{figure}
            \centering
            \includegraphics[width=0.4\textwidth]{images/multimetro/resistori.jpg}
            \hspace{0.05\textwidth}
            \includegraphics[width=0.4\textwidth]{images/multimetro/resistore.jpg}
            \caption{A sinistra alcuni dei \num{50} resistori da \SI{820}{\ohm}. A destra un dettaglio dove è visibile il codice colore.}
            \label{fig:mul:resistore}
        \end{figure}

        Ho effettuato le misure impostando il multimetro in modalità \emph{ohm}, alla portata di \SI{2}{\kilo\ohm}, poggiando i puntali sui terminali di ciascun resistore e aspettando di volta in volta che la lettura si stabilizzasse. I dati raccolti sono riportati in ordine crescente in \tabref{tab:mul:resistori}.
        \begin{table}
            \centering
            \begin{tabular}{cccccccccc}
    \hline
    \multicolumn{10}{c}{Resistenze (\unit{\ohm})}\\\hline\hline
    797 & 806 & 806 & 807 & 807 & 807 & 807 & 807 & 807 & 807 \\
    807 & 808 & 808 & 808 & 808 & 808 & 808 & 808 & 808 & 808 \\
    808 & 808 & 808 & 809 & 809 & 809 & 809 & 809 & 809 & 809 \\
    809 & 809 & 809 & 809 & 809 & 809 & 810 & 810 & 810 & 810 \\
    810 & 810 & 810 & 811 & 811 & 812 & 812 & 812 & 812 & 813 \\ \hline
\end{tabular}
            \caption{Misure di resistenza effettuate su \num{50} resistori distinti.}
            \label{tab:mul:resistori}
        \end{table}

        \subsection{Considerazioni preliminari}
            Notiamo subito che la resistenza media è $R_\text{m} = \SI{808.6}{\ohm}$ con una deviazione standard di $\sigma = \SI{2.3}{\ohm}$; l'errore sul valor medio è quindi $\sigma_R = \sigma / \sqrt{N-1} = \SI{0.33}{\ohm}$, che è confrontabile con la sensibilità dello strumento $\delta R_\text{s} = \SI{1}{\ohm}$.
            
            Questi valori rientrano completamete nell'intervallo fornito dal costruttore; tuttavia, il fatto che tutte le misure siano inferiori a \SI{820}{\ohm} suggerisce la presenza di un errore sistematico.\footnote{Se si trattasse di errori casuali dovuti a imprecisioni di fabbricazione, mi aspetterei letture sia al di sopra che al di sotto del valore di riferimento; è poco probabile che tutte le resistenze devino dal valore teorico allo stesso modo a meno che non si sia verificato un evento che ha alterato tutte le resistenze---un lotto prodotto con lo stesso materiale meno resistente, seppur entro il margine del \SI{5}{\%}, o deterioramento nel tempo.}

            Il dato di resistenza minima di \SI{797}{\ohm} può essere scartato secondo il cirerio di Chauvenet. Esso dista più di $4\sigma$ dal valor medio (ca. $\num{4.17}\sigma$) e il numero di dati atteso\footnote{Per il calcolo di questa probabilità ho fatto riferimento a [INSERIRE TAYLOR!!!]} su un campione di $N = 50$ elementi a una distanza maggiore o uguale a $4\sigma$ è pari a $\num{0.003} \ll 1/2$. Scartando questo dato la nuova media e la nuova deviazione standard sono:
            \begin{equation*}
                R_\text{m} = \SI{808.8}{\ohm}
                \qquad
                \sigma = \SI{1.633}{\ohm}
                \myperiod
            \end{equation*}
            La nuova incertezza sul valor medio è $\sigma_R = \SI{0.24}{\ohm} \lesssim \delta R_\text{s} = \SI{1}{\ohm}$, che è ancora confrontabile con la sensibilità dello strumento. Se tuttavia consideriamo la somma in quadratura delle due incertezze troviamo
            \begin{equation*}
                \overline{\sigma}
                = \sqrt{\sigma_R^2 + \delta R_\text{s}^2}
                = \SI{1.02}{\ohm}
                \simeq \SI{1}{\ohm}
                \mycomma
            \end{equation*}
            per cui assumo $\overline{\sigma} = \SI{1}{\ohm}$ come incertezza sul valor medio.

        \subsection{Test del $\chi^2$}
            Supponiamo che le misure seguano, con una significatività $\alpha = \num{0.05}$, la distribuzione normale centrata in $R_\text{m}$ e di ampiezza $\sigma$:
            \begin{equation*}
                N\pqty{x; R_\text{m}, \sigma}
                = \frac{1}{\sqrt{2\pi\sigma^2}} \exp[ -\frac{1}{2}\pqty{\frac{x - R_\text{m}}{\sigma}}^2]
                \myperiod
            \end{equation*}
            
            Costruiamo quindi un istogramma dei dati. Visto l'intervallo contenuto in cui le misure variano, ho scelto di raccogliere i dati in bin di ampiezza \SI{1}{\ohm}, uno per ciascun valore misurato; ciascun bin si estende da mezza unità \emph{prima} del valore di interesse a mezza unità \emph{dopo}. In \tabref{tab:mul:bin-istogramma} sono riportati i bin e le frequenze osservate $O_k$.
            \begin{table}
                \centering
                \begin{tabular}{ccccc|rc}\hline
    \multicolumn{5}{c}{Intervalli}                  & $O_k$    & $E_k$       \\
    \hline\hline
                  &     & $R$ & $<$ & $\num{806.5}$ & \num{2}  & \num{4.133} \\
    $\num{806.5}$ & $<$ & $R$ & $<$ & $\num{807.5}$ & \num{8}  & \num{5.735} \\
    $\num{807.5}$ & $<$ & $R$ & $<$ & $\num{808.5}$ & \num{12} & \num{6.925} \\
    $\num{808.5}$ & $<$ & $R$ & $<$ & $\num{809.5}$ & \num{13} & \num{7.278} \\
    $\num{809.5}$ & $<$ & $R$ & $<$ & $\num{810.5}$ & \num{6}  & \num{6.655} \\
    $\num{810.5}$ & $<$ & $R$ & $<$ & $\num{811.5}$ & \num{2}  & \num{5.297} \\
    $\num{811.5}$ & $<$ & $R$ & $<$ & $\num{812.5}$ & \num{4}  & \num{3.668} \\
    $\num{812.5}$ & $<$ & $R$ &     &               & \num{1}  & \num{2.211} \\
    \hline
\end{tabular}
                \caption{Suddivisione dei dati per il test del $\chi^2$. Ometto le unità di misura per chiarezza espositiva e semplicità dei calcoli.}
                \label{tab:mul:bin-istogramma}
            \end{table}

            Nella stessa tabella sono riportati i valori attesi $E_k$ per ciascun bin, calcolati moltiplicando la dimensione del campione $N = 49$ per l'integrale di ciascun intervallo della gaussiana. Ho ottenuto gli intervalli convertendo gli estremi in variabili normali standardizzate e ho ricavato l'integrale attraverso un foglio di calcolo.

            È adesso possibile calcolare il $\chi^2$ per definizione:
            \begin{equation*}
                \chi^2
                = \sum_{k=1}^{8} \frac{(O_k - E_k)^2}{E_k}
                = \num{12.98}
                \myperiod
            \end{equation*}
            essendo $d = n - c = 6$ il numero di gradi di libertà, $n = 8$ il numero di bin e $c = 2$ il numero di parametri stimati---media e deviazione standard. Il valore critico per il test del $\chi^2$ è $\chi^2_\text{crit}= \num{12.59}$ \cite{chi2-table}, posso quindi rigettare l'ipotesi nulla che le misure seguano la distribuzione gaussiana.

    \section{Altre misure di resistenze}
            Riporto altre misure eseguite con il multimetro su vari materiali.
            
            \subsection{Resistenze in parallelo}
                Usando $n = 1, 2, 3$ resistori scelti casualmente tra quelli studiati al punto \Sref{s:mul:resistori}, ho misurato la resistenza equivalente $R_\text{o}$ dei resistori montati in parallelo su una breadboard come in \figref{fig:mul:res-parallelo}. I risultati sono riportati in \tabref{tab:mul:res-parallelo}, insieme ai valori teorici, calcolati a partire dalla $R_\text{m}$ del punto precedente come
                \begin{equation*}
                    \frac{1}{R_\text{e}}
                    = \frac{n}{R_\text{m}}
                    \iff R_\text{e} = \frac{R_\text{m}}{n}
                    \mycomma
                \end{equation*}
                e alle incertezze $\delta R_\text{s}$ e $\delta R_\text{e}$. La prima è ancora la sensibilità dello strumento, mentre la seconda è calcolata propagando l'incertezza $\overline{\sigma}$ sul valor medio $R_\text{m}$ trovata al punto \Sref{s:mul:resistori}:
                \begin{equation*}
                    \delta R_\text{e}
                    = \abs{\pdv{R_\text{e}}{R}}\overline{\sigma}
                    = \frac{\overline{\sigma}}{n}
                    \mycomma
                \end{equation*}
                \begin{figure}
                    \centering
                    \includegraphics[width=0.4\textwidth]{images/multimetro/res-parallelo.png}
                    \caption{Illustrazione rappresentativa dei resistori montati in parallelo su una breadboard.}
                    \label{fig:mul:res-parallelo}
                \end{figure}
                \begin{table}
                    \centering
                    \begin{tabular}{ccccc}
    \hline
    $n$ & $R_\text{o}$   & $\delta R_\text{s}$  & $R_\text{e}$  & $\delta R_\text{e}$ \\
    \hline\hline
    $1$ & \num{815}      & \num{1}              & \num{808.8}   &  \num{1}            \\
    $2$ & \num{410}      & \num{1}              & \num{404.4}   &  \num{0.5}          \\
    $3$ & \num{272}      & \num{1}              & \num{269.6}   &  \num{0.3}          \\
    \hline
\end{tabular}
                    \caption{Resistenze equivalenti misurate su resistori in parallelo e relative incertezze. Tutti i valori sono espressi in \unit{\ohm}.}
                    \label{tab:mul:res-parallelo}
                \end{table}

                Osservando i dati si nota che la resistenza equivalente osservata $R_\text{o}$ è sempre maggiore della resistenza teorica $R_\text{e}$. Questo può essere dovuto alla presenza di una resistenza di contatto con la breadboard, che non ho considerato nel calcolo di $R_\text{e}$.

            \subsection{Resistività di un anello}
                Ho misurato la resistenza di un anello metallico che suppongo essere in argento. L'anello è di forma circolare ma presenta un'apertura in basso, per cui ho effettuato la misura della resistenza poggiando i puntali del multimetro in corrispondenza dei punti estremi dell'apertura, come in \figref{fig:mul:anello}. Ho posto il multimetro alla portata di \SI{200}{\ohm} disponendo di una sensibilità $\delta R_\text = \SI{0.1}{\ohm}$; la resistenza misurata è quindi $R = \SI{0.5}{\ohm}$.

                Dalla seconda legge di Ohm possiamo ricavare la resistività del materiale che compone l'anello, supponendo che esso sia omogeneo e isotropo. La resistività è definita come
                \begin{equation}
                    \rho = R\frac{S}{L}
                    \mycomma
                    \label{eq:mul:ohm-law}
                \end{equation}
                dove $S$ è la sezione dell'anello e $L$ la lunghezza equivalente ottenuta rettificando la circonferenza.

                L'anello ha una raggio medio di $r = \SI{2.1(1)}{\centi\meter}$, uno spessore di $d = \SI{0.3(1)}{\centi\meter}$ e un'altezza pari a $h = \SI{0.8(1)}{\centi\meter}$. La sezione dell'anello è quindi $S = hd = \SI{0.24(11)}{\centi\meter\squared}$, mentre, denotando con $\ell = \SI{1.0(1)}{\centi\meter}$ la distanza che separa i due puntali e che quindi non partecipa alla resistenza, la lunghezza equivalente è $L = 2\pi r - \ell = \SI{12.3(7)}{\centi\meter}$.

                Dal momento che l'anello è formato da fili cilindrici intrecciati, la sezione $S$ rappresenta in realtà una sovrastima della sezione effettiva. Il rapporto tra la superficie di un cerchio di raggio $d/2$ e quella di un quadrato di lato $d$ è pari a $\pi / 4$, per cui la sezione effettiva è $S_\text{eff} = S \pi / \num{4} = \SI{0.19(8)}{\centi\meter\squared}$.
                
                La resistività dell'anello calcolata dalla \eqref{eq:mul:ohm-law} è quindi
                \begin{equation*}
                    \rho = R\frac{S_\text{eff}}{L} = \SI{0.0077(55)}{\ohm\centi\meter} = \SI{7.7(55)e-5}{\ohm\meter}
                    \mycomma
                \end{equation*}
                valore che si discosta di diversi ordini di grandezza da quello noto di \SI{1.59e-8}{\ohm\meter} \cite{Griffiths2012-qz}. Un risultato così elevato può essere attribuito sia all'incertezza sulle misure geometriche, sia al fatto che l'anello non sia composto da argento puro, sia alla presenza di ossidazione superficiale o di cattivi contatti tra i fili intrecciati che ne aumentano la resistenza complessiva.
                \begin{figure}
                    \centering
                    \includegraphics[width=0.4\textwidth]{images/multimetro/anello.jpg}
                    \caption{Fotografia dell'anello. I tratti in rosso indicano il punti in cui sono stati poggiati i puntali del multimetro.}
                    \label{fig:mul:anello}
                \end{figure}



        \appendix
        \chapter{Codice per la formula di Bethe--Bloch}\label{app:bet}
    Riporto in questa appendice il codice utilizzato per l'esperienza descritta al Capitolo \Sref{ch:bet}. Come descritto nel capitolo dedicato, il programma simula il passaggio di particelle $\alpha$ attraverso un foglio di alluminio spesso \SI{0.018}{\milli\meter}.
    \lstinputlisting[language = C, escapechar = |]{code/bethe.txt}
        \chapter{Codice per Arduino}\label{app:ard}
    Riporto in questa appendice il codice utilizzato per l'esperienza descritta al Capitolo \Sref{ch:ard}.
    \lstinputlisting[language = C++, escapechar = |]{code/arduino-temp.ino}

    \backmatter
        \printbibliography  

\end{document}
