\begin{tabular*}{\textwidth}{@{\extracolsep{\fill}}lll}\hline\rule{0pt}{8pt}%
    Simbolo & Definizione & Valore o unità di misura\\[0.5pt]
    \hline\hline\rule{0pt}{9pt}%
    $m_e c^2$ & massa a riposo dell'elettrone $\times c^2$ & \SI{ 0.51099895000(15)}{\mega\eV}\\
    $r_e$ & raggio classico dell'elettrone $ e^2/4\pi \epsilon_0 m_e c^2$ & \SI{2.8179403227(19)}{\femto\meter}\\
    $N_\text{A}$ & numero di Avogadro & \SI{ 6.02214076e+23}{mol^{-1}}\\[2pt]
    \hline\rule{0pt}{9pt}%
    $\rho$ & densità & \unit{\g\,\centi\meter^{-3}}\\
    $x$ & massa per unità di area & \unit{\gram\,\centi\meter^{-2}}\\ 
    $M$ & massa della particella incidente & \unit{\mega\eV \,\mathit{c}^{-2}}\\
    $E$ & energia della particella incidente $\gamma M c^2$ & \unit{\mega\eV}\\
    $W_\text{max}$ & massima energia trasferibile per collisioni & \unit{\mega\eV}\\
    $z$ & numero di carica della particella incidente & \\
    $Z$ & numreo atomico del bersaglio & \\
    $A$ & numero di massa atomica del bersaglio & \\
    $K$ & $4\pi N_\text{A} r_e^2 m_e c^2$ & \SI{0.307075}{\mega\eV\,\mol^{-1}\,\centi\meter^2}\\
    $I$ & energia media di eccitazione & \unit{\eV}\\
    $\delta\pqty{\beta\gamma}$ & correzione di ionizzazione & \\[1pt]
    \hline
\end{tabular*}